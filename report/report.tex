\documentclass[a4paper,10pt]{article}
\usepackage[french]{babel}
\usepackage[utf8]{inputenc}
\usepackage[left=2.5cm,top=2cm,right=2.5cm,nohead,nofoot]{geometry}
\usepackage{url}
\usepackage[T1]{fontenc}
\usepackage{float}
\usepackage{afterpage}
\usepackage{amsmath}
\usepackage{graphicx}
\usepackage{tabularx}
\usepackage{csquotes}
\usepackage{fullpage}
\usepackage{pdfpages}
\usepackage{listings}
\usepackage{color}

\usepackage{listings}
\usepackage{mdframed}
\usepackage{color}
\usepackage{pdflscape}

\definecolor{mygreen}{rgb}{0,0.6,0}
\definecolor{mygray}{rgb}{0.5,0.5,0.5}
\definecolor{mymauve}{rgb}{0.58,0,0.82}

\lstset{ %
  backgroundcolor=\color{white},   % choose the background color; you must add \usepackage{color} or \usepackage{xcolor}
  basicstyle=\footnotesize,        % the size of the fonts that are used for the code
  breakatwhitespace=false,         % sets if automatic breaks should only happen at whitespace
  breaklines=true,                 % sets automatic line breaking
  captionpos=b,                    % sets the caption-position to bottom
  commentstyle=\color{mygreen},    % comment style
  deletekeywords={...},            % if you want to delete keywords from the given language
  escapeinside={\%*}{*)},          % if you want to add LaTeX within your code
  extendedchars=true,              % lets you use non-ASCII characters; for 8-bits encodings only, does not work with UTF-8
  frame=single,                    % adds a frame around the code
  keepspaces=true,                 % keeps spaces in text, useful for keeping indentation of code (possibly needs columns=flexible)
  keywordstyle=\color{blue},       % keyword style
  language=Octave,                 % the language of the code
  otherkeywords={*,...},           % if you want to add more keywords to the set
  numbers=left,                    % where to put the line-numbers; possible values are (none, left, right)
  numbersep=5pt,                   % how far the line-numbers are from the code
  numberstyle=\tiny\color{mygray}, % the style that is used for the line-numbers
  rulecolor=\color{black},         % if not set, the frame-color may be changed on line-breaks within not-black text (e.g. comments (green here))
  showspaces=false,                % show spaces everywhere adding particular underscores; it overrides 'showstringspaces'
  showstringspaces=false,          % underline spaces within strings only
  showtabs=false,                  % show tabs within strings adding particular underscores
  stepnumber=2,                    % the step between two line-numbers. If it's 1, each line will be numbered
  stringstyle=\color{mymauve},     % string literal style
  tabsize=2,                     % sets default tabsize to 2 spaces
  title=\lstname                   % show the filename of files included with \lstinputlisting; also try caption instead of title
}

\usepackage[pdftex,
            pdfauthor={A. Caccia, A. Madeira Cortes},
            pdftitle={Informatique Fondamentale - Projet},
            pdfsubject={INFO-F-302 : Informatique Fondamentale - Projet}]{hyperref}

\linespread{1.1}

\setlength{\parskip}{0.5em}

\def\ojoin{\setbox0=\hbox{$\bowtie$}%
  \rule[-.02ex]{.25em}{.4pt}\llap{\rule[\ht0]{.25em}{.4pt}}}
\def\leftouterjoin{\mathbin{\ojoin\mkern-5.8mu\bowtie}}
\def\rightouterjoin{\mathbin{\bowtie\mkern-5.8mu\ojoin}}
\def\fullouterjoin{\mathbin{\ojoin\mkern-5.8mu\bowtie\mkern-5.8mu\ojoin}}

\begin{document}

\begin{titlepage}
    \begin{center}
        \textbf{\textsc{Université Libre de Bruxelles}}\\
        \textbf{\textsc{Faculté des Sciences}}\\
        \textbf{\textsc{Département d'Informatique}}

        \vfill{}
        \vfill{}

        \begin{center}
            {\Huge INFO-F-302 : Informatique Fondamentale}
        \end{center}

        {\Huge \par}

        \begin{center}
            {\LARGE Projet : Modélisation du problème du orthogonal packing et résolution avec l’outil MiniSat}
        \end{center}

        {\Huge \par}

        \begin{center}
            {\large A. Caccia, A. Madeira Cortes}
        \end{center}

        {\Huge \par}
        \vfill{}
        \vfill{}

        {\large\par}
        \vfill{}
        \vfill{}
        % \enlargethispage{3cm} % do not remove

        \textbf{Année académique 2015--2016}
    \end{center}
\end{titlepage}

\tableofcontents
\newpage

\section{Énoncé}

\section{Questions}

\subsection{Question 1: Formalisation des contraintes que doit satisfaire $\mu$}

\subsection{Question 2: Construction d'une formule en FNC telle que celle-ci est SAT ssi il existe $\mu$ correcte}

\subsection{Question 3: Implémentation et tests sur des exemples simples}

\subsection{Question 4: Si R est un carré, trouver la plus petite dimension de celui-ci admettant une solution pour des rectangles donnés}

\subsection{Question 5: Si R est un carré et n un entier donné, trouver la plus petite dimension de R admettant une solution pour des carrés de taille $i \times i$ pour tout $i \leq n$}

\subsection{Question 6: Ajout d'une troisième dimension}

\subsection{Question 7: Contrainte - ne pas faire flotter des pavés droits dans l'espace}

\section{Questions bonus}

\subsection{Question 8: Pivotage d'un des rectangles}

\subsection{Question 9: Introduction d'un minimum de contact entre les rectangles et les bords de R}

\subsection{Question 10: Maximisation de ce contact avec MAX-SAT}

\end{document}
